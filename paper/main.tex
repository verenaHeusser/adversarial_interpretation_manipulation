%%
%% This is file `sample-sigconf.tex',
%% generated with the docstrip utility.
%%
%% The original source files were:
%%
%% samples.dtx  (with options: `sigconf')
%% 
%% IMPORTANT NOTICE:
%% 
%% For the copyright see the source file.
%% 
%% Any modified versions of this file must be renamed
%% with new filenames distinct from sample-sigconf.tex.
%% 
%% For distribution of the original source see the terms
%% for copying and modification in the file samples.dtx.
%% 
%% This generated file may be distributed as long as the
%% original source files, as listed above, are part of the
%% same distribution. (The sources need not necessarily be
%% in the same archive or directory.)
%%
%% The first command in your LaTeX source must be the \documentclass command.
\documentclass[sigconf]{acmart}

%%
%% \BibTeX command to typeset BibTeX logo in the docs
\AtBeginDocument{%
  \providecommand\BibTeX{{%
    \normalfont B\kern-0.5em{\scshape i\kern-0.25em b}\kern-0.8em\TeX}}}

%% Rights management information.  This information is sent to you
%% when you complete the rights form.  These commands have SAMPLE
%% values in them; it is your responsibility as an author to replace
%% the commands and values with those provided to you when you
%% complete the rights form.
% \setcopyright{acmcopyright}
% \copyrightyear{2020}
% \acmYear{2020}
% \acmDOI{10.1145/1122445.1122456}

%% These commands are for a PROCEEDINGS abstract or paper.
\acmConference[Explainability '20]{Explainability '20: ACM Symposium on Neural Model Explanations}{December 16--20, 2020}{Karlsruhe, DE}
\acmBooktitle{Explainability '20: ACM Symposium on Neural Model Explanations,
December 16--20, 2020, Karlsruhe, DE}
% \acmPrice{15.00}
% \acmISBN{978-1-4503-XXXX-X/18/06}

\usepackage{todonotes}

\begin{document}

%%
%% The "title" command has an optional parameter,
%% allowing the author to define a "short title" to be used in page headers.
% \title{Manipulating Model Explanations: How to fool what tries to make sense of}
\title{Manipulating Model Explanations: Why you shouldn't trust me}



\author{Verena Heusser}
\email{verena.heusser@student.kit.edu}
\affiliation{%
  \institution{Karlsruhe Institute of Technology (KIT)}
  \city{Karlsruhe}
  \country{Germany}
}


\begin{abstract}
  This paper reviews state-of-the-art approaches to model explanations with a focus on those techniques that try to 
  fool these methods. 

\end{abstract}

%%
%% The code below is generated by the tool at http://dl.acm.org/ccs.cfm.
%% Please copy and paste the code
%%
\begin{CCSXML}
  <ccs2012>
  <concept>
  <concept_id>10010147.10010257.10010293.10010294</concept_id>
  <concept_desc>Computing methodologies~Neural networks</concept_desc>
  <concept_significance>500</concept_significance>
  </concept>
  </ccs2012>
\end{CCSXML}

\ccsdesc[500]{Computer systems organization~Embedded systems}
\ccsdesc[300]{Computer systems organization~Redundancy}
\ccsdesc{Computer systems organization~Robotics}
\ccsdesc[100]{Networks~Network reliability}

%%
%% Keywords. The author(s) should pick words that accurately describe
%% the work being presented. Separate the keywords with commas.
\keywords{Interpretability, Neural networks, adversarial training}

%% A "teaser" image appears between the author and affiliation
%% information and the body of the document, and typically spans the
%% page.

\maketitle


\section{Introduction}
\label{sec:introduction}
% questioning the how

In recent years deep learning models have demonstrated superior performance in a number of tasks. While the performance is still rising and more task domains are accomplished, these models still remain black boxes often uninterpretable even by experts. 
In many domains, neural networks currently are the state-of-the-art solution. However, their superior performance comes at the cost of complexity, as the models often employ millions to even billions of parameters in order to achieve universal function approximation. This complexity means a drawback in interpretability as the decision making process of such a network cannot be followed by humans without the help of further tools. For instance, withing object recognition one would like to assume that the presence (or absence) of an object in the image causes a model to decide for a specific object category, closely akin to how humans base their decision process. 

At present, many concerns regard the coherence of automated decisions to ethical standards. Regarding the expanding number of tasks that automated computer algorithms are used for nowadays, this concern is becoming even more important, as machine learning models are moving out of the lab into the real world. The application of algorithms for prediction of recidivism rates are already applied at court \cite{chouldechova2017fair}, and the filtering of job applicants, piloting of self-driving cars, diagnoses of diseases or automated food recognition \cite{ruede2020multi} are already in use.

% sort of prove that neural networks make a safe decision without biases

% much broader context to create security and large scale applications 

% https://dl.acm.org/doi/pdf/10.1145/3387514.3405859?casa_token=lCc16GOTZsEAAAAA:gypLNU1o2Wwl3wt_b8stRbb0mgxEomX8PWprPeciNdkhVften3-5E01RM50e0W9NGQaGd4TrLOhA

% explainability can be viewed as an active characteristic of a model, denoting any action orprocedure taken by a model with the intent of clarifying or detailing its internal functions.


Thus, automated interpretation methods are required to make sense of the reasoning process and stability of such deep learning based models and to ensure that a model makes decisions without unfair or hidden biases. 
The research field approaching the explanation or validation of machine learning models is called explainable artificial intelligence (XAI).
%  not knowing why models decide
Not knowing about the biases of a network the vastly advancing technology of machine learning to be used in high-stakes and safety critical applications and prevent real-life deployment of such systems. 
Furthermore, the rise in machine learning model deployments also caused the development of adversarial attacks. These attacks attempt to fool a machine learning model by providing deceptive input. Fooling refers to the resulting malfunction of the model. 
% TODO first example

Not knowing about attacks and data arranged to exploit specific vulnerabilities has contributed to a relatively new research field of XAI comprising topics of (1) \textit{model explanations}, (2) \textit{adversarial attacks}, or manipulation methods and (3) the field of \textit{adversarial manipulations of model explanations}. All of this is also known by the name of robust machine learning or even explainable artificial intelligence, as all subfields have the common goal to make models more robust and safe for deployment. 
(1) refers to the development of techniques that can be used to understand and explain the decision making process of a machine learning model or even the development of models that are inherently interpretable. (2) is the field of detecting vulnerabilites in models that cause models to be deceived by altered input. 
(3) is the main topic of this paper, i.e. how to fool explanation models in order to detect vulnerabilities and malfunctions in explanation methods. 

Detecting such vulnerabilities in models is most crucial 

Fragility limits how much wecan trust and learn from the interpretations



% While most of the approaches to explainability focus on the application to computer vision tasks, other domains are seldomly chosen. 
Most research on explainability focuses on the application of computer vision tasks. 
% todo cite a lot here
% https://www.bmc.com/blogs/machine-learning-interpretability-vs-explainability/ 
Most works in the field of XAI focus on image classification task, mostly because visualizations of a neural networks prediction can be easily verified by a human. The general purpose of image classification is to detect what objects are in an image. If a model works can be checked rather easily (if an image contains a cat, the prediction of a neural network should be cat and not some other object category). However, how it works (\textit{interpretability}), i.e. based on which features in the image the decision is made or which parameters in the model influence the prediction most, is an entirely different matter (\textit{explainability}).  


More importantly, while a big motivation for the development of robust and explainable systems is to overcome biases in models, datasets with 
direct implication of biases are seldomly used and by far not treated as benchmarking scenarios for explainability analyses.  


% Theoretical background
% why do adv attacks make sense? most models are trained on iid samples and thus not directly applicable to the real world, as the real world violates this statistical assumption. 




% Outline
The overview presented in this article examines the existing literature and contributions in the field of XAI focusing on methods to manipulate explanation methods.  
The critical literature analysis might serve as a motivation and step towards the biggest problems in XAI: How to make sure that interpretations of models are truly valid. 
This paper is structured as follows... 


\section{Manipulation of explanations}

\subsection{Explanation Methods}
\label{subsec:explanation_methods}
Explanation models aim at making complex and inherently uninterpretable black box models interpretable by creating human readable visualizations. 
A frequently used type of explanation methods are feature attributions mapping a each input feature to a model to a numeric score. This score should quantify the importance of the feature relative to the model output. The resulting attribution map is then visualized as a heatmap projected onto the input sample to interpret the input attributes regarding which ones are the most helpful for forming the final prediction. 



\subsubsection{Explanation methods using gradient information}





\subsection{Manipulation Methods}
\label{subsec:manipulation_methods}
Most of the explanation methods outlined in sec. \autoref{subsec:explanation_methods} have been shown to be vulnerable to adversarial perturbations. 
Manipulation methods often show that there exist small feature changes resulting in a change of the explanation methods output while the output of the model itself does not change. 




Most approaches aim at providing a relevance measure of the input features. 


\subsubsection{Input Manipulations}
\label{subsec:input_manipuls}

The general approach is to perturb input data while observing the effect of this perturbation. As found in TODO, visually-imperceptible perturbations of an input image can make explanations worse for the same model and interpreter. 




\subsubsection{Model Manipulations}
Contrary to the methods introduced in sec. \autoref{subsec:input_manipuls}, the methods in this section do not operate on the input space of models but rather on the model parameter space itself. 
As first introduced by Heo et al. \cite{fooling_nn_interpreters} in 2017, this line of research is comparably new. The authors find that perturbed model parameters can also make explanations worse for the same input images and interpreters. 


\subsubsection{Transferability of Manipulations}

\section{Characterization of Robustness}
\label(sec:robustness)

This section introduces common evaluation strategies designed to test the 
robustness of either model or interpreter towards an applied attack. 

TODO: 
- auch auf entlarvungsmethoden eingehen? 



\section{Transferability of Perturbations}
\label(sec:transferability)

input perturbations do not propagate to the whole validation set. 
On the contrary, model manipulations are non-local perturbations, meaning that they do not merely perturb an input sample
but rather effect all samples in the way that the model itself is changed. 


\section{Experiments}
\label{sec:experiments}
In this section, several experiments are evaluated that were conducted to replicate findings of other studies. Furthermore these approaches are extended to other domains and datasets. 


\subsection{Explanation Methods}

\subsection{Manipulation Methods}

\subsection{Models}

\subsection{Datasets}

\subsubsection{ImageNet}
\subsubsection{Recidivism Dataset}
\subsubsection{German dataset of..}

\section{Discussion}
\label{sec:discussion}

\footnote{All figures in this article are produced by the author unless noted otherwise.}

% we replicated the findings of the presented manipulation methods from \cite{fooling_nn_interpreters} and.. 

This paper summarizes the current approaches to manipulating model interpretation methods. 
On the one hand, the findings suggest that our models aro not fully aligned with how human information processing works. If machine learning interpretation models would decide by the criteria we humans employ for tasks such as image classification, there would be no fooling of interpretation models by input or model manipulations. 
On the other hand, it was shown that advances in machine learning models has led to models that rely too much on the data they are trained on (the i.i.d. assumption), thus showing a high susceptibility to o.o.d. properties or properties that are highly correlated with labels in the dataset but are not distinctive in the real world (such as image backgrounds). Models and interpreters can still be misled in a large and systematic manner. 

% However, we hope that the vastly expanding and progressing field of XAI will help to move towards more robust, reliable ond human-aligned machine learning models. 

While there exists a number of review papers on XAI and it's various subfields, this report is to the best of our knowledge the first one to comprehensively review manipulation methods for interpreters. 

A growing number of studies gives evidence for model how interpretation methods can be gamed. Among these are the studies outlined in \autoref{sec:manipulations}. Other studies also raise concerns about if standard deep learning practices are valid, such as the work on fooling the broadly used attention mechanism \cite{jain2019attention}. % http://lcfi.ac.uk/media/uploads/files/DimanovBhattJamnikWeller_YouShouldntTrustMe.pdf how  attention-based  methods  could  be  fooled.  (Jainand  Wallace  2019)  showed  that  ‘attention  is  not  explana-tion’, demonstrating that attention maps could be manipu-lated after training without altering predictions

We believe that identifying risks and adversaries helps to open up research on more robust interpretation methods. 

% Interpretation methods can be categorized based on if they maintain local consistency among explanations (i.e. finding an explanation that is true for single data samples and their neighbors) or based on if they try to find global explanations, being true for all samples of a class. 
% As there exist model manipulations methods, that structurally alter the models by adapting tre loss function, this line of global model fooling though being approached is still in its infancy. 


\subsection{Conclusion}

% We believe our algorithms can facilitate developingmore robust network interpretation tools that truly explainthe network’s underlying decision making process. https://openaccess.thecvf.com/content_ICCV_2019/papers/Subramanya_Fooling_Network_Interpretation_in_Image_Classification_ICCV_2019_paper.pdf 
% --> making nns more robust to adversarial attacks might also benefit the robustness of fooling methods?

Finally, it must be noted that the suitability of a method depends on its application domain. 


Much critique has been applied to methods aiming at interpreting complex and potentially non-interpretable models in the domain of computer vision. Some researchers argue it is not worthwhile to study non interpretable systems while dismissing that using inherently interpretable models in the first place might be the better approach. 

Adversarial attacks show that machine learning systems are still fundamentally fragile: They may be successful in a number of tasks, but fail to adapt to o.o.d. scenarios, i.e. when being applied to unfamiliar territory. 
% Our results raise concerns on how interpretations of neural networks can be manipulated.
% fail unpredictably

 The findings about manipulating interpretations do not suggest that interpretations are completely meaningless, just as adversarial attacks on predictions models do not imply that machine learning models are useless. However, they suggest that there still are fundamental flaws in the way neural networks operate und that much caution and supervision sould be applied if they are to be deployed in the real world. 

This paper follows the footsteps of \cite{lipton2018mythos}, trying to caution against blindly putting faith into post-hoc explanation methods. Moreover, we propose that checking the robustness of interpretation methods not only with respect to adversarial input manipulations but also with respect to adversarial model manipulation should be a necessary proof of concept. 


% We argue checking the robustness of interpretation methods with respect to our adversarialmodel manipulation should be an indispensable criterion for the interpreters in addition to the sanitychecks proposed in [27];


%%%%%%%%%%%%% FUTURE WORK
\subsection{Future Work}
We see several possible future directions of future work. 
Firstly, for approaching the discrepency of in-lab and real-life applications of, more focus might be laid in the development better performance metrics for both measuring the performance of machine learning models as well as their interpreters. 
More specifically, it might be fruitful to further investigate the correlation between ood samples and the performance of an interpretation method. So far, most of these findings are limited to specific experimental settings. 


There is also no work proposing a benchmarking for ... 

What is currently sparse is the comparison between different interpretation techniques and the relation of interpretation fragility to the model class, interpretation method and the task type and dataset structure. 


The output of the interpretation method is projected onto the original image for better human readability
--> dangerous to trust this


% % Visually appealing methods and easiy visual assessment of results. 
% Most works in the field of XAI focus on image classification tasks, mostly because visualizations of a neural networks prediction can be easily verified by a human. The general purpose of image classification is to detect what objects are in an image. If a model works can be checked rather easily (if an image contains a cat, the prediction of a neural network should be cat and not some other object category). However, how it works (\textit{interpretability}), i.e. based on which features in the image the decision is made or which parameters in the model influence the prediction most, is an entirely different matter (\textit{explainability}).  

% More importantly, while a big motivation for the development of robust and explainable systems is to overcome biases in models, datasets with direct implication of biases are seldomly used and by far not treated as benchmarking scenarios for explainability analyses.  

\section{Conclusion}
Finally, it must be noted that the suitability of a method depends on its application domain. 


Much critique has been applied to methods aiming at interpreting complex and potentially non-interpretable models. 
Some researchers argue it is not worthwhile to study non interpretable systems while dismissing that using inherently interpretable models in the first place might be the better approach. 

Adversarial attacks show that machine learning systems are still fundamentally fragile: They may be successful in a number of tasks, but fail to adapt to ood scenarios, i.e. when being applied to unfamiliar territory. 
% Our results raise concerns on how interpretations of neural networks can be manipulated.
% fail unpredictably

 The findings about manipulating interpretations do not suggest that interpretations are completely meaningless, just as adversarial attacks on predictions models do not imply that machine learning models are useless. However, they suggest that there still are fundamental flaws in the way neural networks operate und that much caution and supervision sould be applied if they are to be deployed in the real world. 



% \balance
\bibliography{mybib}{}
% \bibliographystyle{plain}
\bibliographystyle{ACM-Reference-Format}

%%
%% If your work has an appendix, this is the place to put it.
% \appendix

\end{document}
\endinput