\section{Discussion}
\label{sec:discussion}

\footnote{All figures in this article are produced by the author unless noted otherwise.}

% we replicated the findings of the presented manipulation methods from \cite{fooling_nn_interpreters} and.. 

This paper summarizes the current approaches to manipulating model interpretation methods. 
On the one hand, the findings suggest that our models aro not fully aligned with how human information processing works. If machine learning interpretation models would decide by the criteria we humans employ for tasks such as image classification, there would be no fooling of interpretation models by input or model manipulations. 
On the other hand, it was shown that advances in machine learning models has led to models that rely too much on the data they are trained on (the i.i.d. assumption), thus showing a high susceptibility to o.o.d. properties or properties that are highly correlated with labels in the dataset but are not distinctive in the real world (such as image backgrounds). Models and interpreters can still be misled in a large and systematic manner. 

% However, we hope that the vastly expanding and progressing field of XAI will help to move towards more robust, reliable ond human-aligned machine learning models. 

While there exists a number of review papers on XAI and it's various subfields, this report is to the best of our knowledge the first one to comprehensively review manipulation methods for interpreters. 

A growing number of studies gives evidence for model how interpretation methods can be gamed. Among these are the studies outlined in \autoref{sec:manipulations}. Other studies also raise concerns about if standard deep learning practices are valid, such as the work on fooling the broadly used attention mechanism \cite{jain2019attention}. % http://lcfi.ac.uk/media/uploads/files/DimanovBhattJamnikWeller_YouShouldntTrustMe.pdf how  attention-based  methods  could  be  fooled.  (Jainand  Wallace  2019)  showed  that  ‘attention  is  not  explana-tion’, demonstrating that attention maps could be manipu-lated after training without altering predictions

We believe that identifying risks and adversaries helps to open up research on more robust interpretation methods. 

% Interpretation methods can be categorized based on if they maintain local consistency among explanations (i.e. finding an explanation that is true for single data samples and their neighbors) or based on if they try to find global explanations, being true for all samples of a class. 
% As there exist model manipulations methods, that structurally alter the models by adapting tre loss function, this line of global model fooling though being approached is still in its infancy. 


\subsection{Conclusion}

% We believe our algorithms can facilitate developingmore robust network interpretation tools that truly explainthe network’s underlying decision making process. https://openaccess.thecvf.com/content_ICCV_2019/papers/Subramanya_Fooling_Network_Interpretation_in_Image_Classification_ICCV_2019_paper.pdf 
% --> making nns more robust to adversarial attacks might also benefit the robustness of fooling methods?

Finally, it must be noted that the suitability of a method depends on its application domain. 


Much critique has been applied to methods aiming at interpreting complex and potentially non-interpretable models in the domain of computer vision. Some researchers argue it is not worthwhile to study non interpretable systems while dismissing that using inherently interpretable models in the first place might be the better approach. 

Adversarial attacks show that machine learning systems are still fundamentally fragile: They may be successful in a number of tasks, but fail to adapt to o.o.d. scenarios, i.e. when being applied to unfamiliar territory. 
% Our results raise concerns on how interpretations of neural networks can be manipulated.
% fail unpredictably

 The findings about manipulating interpretations do not suggest that interpretations are completely meaningless, just as adversarial attacks on predictions models do not imply that machine learning models are useless. However, they suggest that there still are fundamental flaws in the way neural networks operate und that much caution and supervision sould be applied if they are to be deployed in the real world. 

This paper follows the footsteps of \cite{lipton2018mythos}, trying to caution against blindly putting faith into post-hoc explanation methods. Moreover, we propose that checking the robustness of interpretation methods not only with respect to adversarial input manipulations but also with respect to adversarial model manipulation should be a necessary proof of concept. 


% We argue checking the robustness of interpretation methods with respect to our adversarialmodel manipulation should be an indispensable criterion for the interpreters in addition to the sanitychecks proposed in [27];


%%%%%%%%%%%%% FUTURE WORK
\subsection{Future Work}
We see several possible future directions of future work. 
Firstly, for approaching the discrepency of in-lab and real-life applications of, more focus might be laid in the development better performance metrics for both measuring the performance of machine learning models as well as their interpreters. 
More specifically, it might be fruitful to further investigate the correlation between ood samples and the performance of an interpretation method. So far, most of these findings are limited to specific experimental settings. 


There is also no work proposing a benchmarking for ... 

What is currently sparse is the comparison between different interpretation techniques and the relation of interpretation fragility to the model class, interpretation method and the task type and dataset structure. 


The output of the interpretation method is projected onto the original image for better human readability
--> dangerous to trust this


% % Visually appealing methods and easiy visual assessment of results. 
% Most works in the field of XAI focus on image classification tasks, mostly because visualizations of a neural networks prediction can be easily verified by a human. The general purpose of image classification is to detect what objects are in an image. If a model works can be checked rather easily (if an image contains a cat, the prediction of a neural network should be cat and not some other object category). However, how it works (\textit{interpretability}), i.e. based on which features in the image the decision is made or which parameters in the model influence the prediction most, is an entirely different matter (\textit{explainability}).  

% More importantly, while a big motivation for the development of robust and explainable systems is to overcome biases in models, datasets with direct implication of biases are seldomly used and by far not treated as benchmarking scenarios for explainability analyses.  