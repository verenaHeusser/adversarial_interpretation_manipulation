\section{Introduction}
\label{sec:introduction}

In recent years deep learning models have demonstrated superior performance in a number of tasks. While the performance is still rising and more task domains are accomplished, these models still remain black boxes often being uninterpretable even by experts. 
In many domains, neural networks currently are the state-of-the-art solution. However, their superior performance comes at the cost of complexity, as the models often employ millions to even billions of parameters in order to achieve universal function approximation. This complexity means a drawback in interpretability as the decision making process of such a network cannot be followed by humans without the help of further tools. For instance, withing object recognition one would like to assume that the presence (or absence) of an object in the image causes a model to decide for a specific object category, closely akin to how humans base their decision process. 

At present, many concerns regard the conformance of automated decisions to ethical standards. Regarding the expanding number of tasks that computer tasks are used for nowadays, this concern is becoming even more valid. The application of algorithms for prediction of recidivism rates as even applied at court \cite{chouldechova2017fair}, the filtering of job applicants or ... is already in use \cite{lipton2018mythos}. 

% sort of prove that neural networks make a safe decision without biases

% much broader context to create security and large scale applications 

% https://dl.acm.org/doi/pdf/10.1145/3387514.3405859?casa_token=lCc16GOTZsEAAAAA:gypLNU1o2Wwl3wt_b8stRbb0mgxEomX8PWprPeciNdkhVften3-5E01RM50e0W9NGQaGd4TrLOhA

Thus, automated interpretation methods are required to make sense of the reasoning process of such deep learning based models and to ensure that a model makes decisions without unfair or hidden biases. 
The research field approaching the explanation or validation of machine learning models decision processes is called explainable artificial intelligence (XAI).
%  not knowing why models decide
Not knowing about the biases of a network the vastly advancing technology of machine learning to be used in high-stakes and safety critical applications and prevent real-life deployment of such systems. 
Furthermore, the rise in machine learning model deployments also caused the development of adversarial attacks. These attacks attempt to fool a machine learning model by providing deceptive input. Fooling refers to the resulting malfunction of the model. 
% TODO first example

Not knowing about attacks and data arranged to exploit specific vulnerabilities has contributed to a relatively new  research field of XAI comprising topics of (1) \textit{model explanations}, (2) \textit{adversarial attacks}, or manipulation methods and (3) the field of \textit{manipulations of model explanations}. All of this is also known by the name of robust machine learning or even explainable artificial intelligence, as all subfields have the common goal to make models more robust and safe for deployment. 
(1) refers to the development of techniques that can be used to understand and explain the decision making process of a machine learning model or even the development of models that are inherently interpretable. (2) is the field of detecting vulnerabilites in models that cause models to be deceived by altered input. 
(3) is the main topic of this paper, i.e. how to fool explanation models in order to detect vulnerabilities and malfunctions in explanation methods. 


While most of the approaches to explainability focus on the application to computer vision tasks, other domains are seldomly chosen. 
More importantly, while a big motivation for the development of robust and explainable systems is to overcome biases in models, datasets with 
direct implication of biases are seldomly used and by far not treated as benchmarking scenarios for explainability analyses.  


% Theoretical background
% why do adv attacks make sense? most models are trained on iid samples and thus not directly applicable to the real world, as the real world violates this statistical assumption. 




% Outline
The overview presented in this article examines the existing literature and contributions in the field of XAI focusing on methods to manipulate explanation methods.  
The critical literature analysis might serve as a motivation and step towards the biggest problems in XAI: How to make sure that interpretations of models are truly valid. 
This paper is structured as follows... 
