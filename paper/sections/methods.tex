\section{Manipulation of explanations}

\subsection{Explanation Methods}
\label{subsec:explanation_methods}
A frequently used type of explanation methods are feature attributions mapping a each input feature to a model to a numeric score. This score should quantify the importance of the feature relative to the model output. The resulting attribution map is then visualized as a heatmap projected onto the input sample to interpret the input attributes regarding which ones are the most helpful for forming the final prediction. 



\subsubsection{Explanation methods using gradient information}





\subsection{Manipulation Methods}
\label{subsec:manipulation_methods}
Most of the explanation methods outlined in sec. \ref{subsec:explanation_methods} have been shown to be vulnerable to adversarial perturbations. 
Manipulation methods often show that there exist small feature changes resulting in a change of the explanation methods output while the output of the model itself does not change. 




Most approaches aim at providing a relevance measure of the input features. 


\subsubsection{Input Manipulations}
\label{subsec:input_manipuls}

The general approach is to perturb input data while observing the effect of this perturbation. As found in TODO, visually-imperceptible perturbations of an input image can make explanations worse for the same model and interpreter. 




\subsubsection{Model Manipulations}
Contrary to the methods introduced in sec. \ref{subsec:input_manipuls}, the methods in this section do not operate on the input space of models but rather on the model parameter space itself. 
As first introduced by Heo et al. \cite{fooling_nn_interpreters} in 2017, this line of research is comparably new. The authors find that perturbed model parameters can also make explanations worse for the same input images and interpreters. 


\subsubsection{Transferability of Manipulations}